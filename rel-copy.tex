





% \section{Related Work}\label{sec:related}
% \sajreview{In this section, we discuss existing work on \vita, data management for data science, and domain specific algebra design.} 

% \stitle{Visual interactive text analytics.}
% \sajreview{Visual text analytics 
% combines the advantages of interactive visualization and
% text mining techniques to facilitate the exploration and analysis of large-scale textual data from both a structured and
% unstructured perspective. Visualization techniques, both  
% basic 
% (\eg scatterplot, line chart, treemap, rose plot) and advanced (\eg wordcloud, steam graph, flow graph, 3D visualizations)~\cite{liu2018bridging}, are applied to numerous uses-cases like review exploration~\cite{zhang2020teddy}, sentiment analysis~\cite{kucher2018state}, text summarization~\cite{carenini2006interactive}. While these work highlight the appeal of integrating interactive visualization with text analysis, they lack the generalizability of \system, which has the potential to guide users in authoring different \vita use-case within a single system.}

% \stitle{Data management for \vita.}
% \sajreview{While a plethora of text analytics tools are being used for various uses cases, more often than not these tools
% use a stack of independent solutions for data storage and
% processing glued together by scripting languages. There are a number of work from the data management that focus on designing systems for scalabale analytics~\cite{modin,arrow,dask}, efficient storage engines for storing and accessing data~\cite{tiledb,raasveldt2020data}. \system is inspired by these projects with specific focus on developing an efficient storage model, enabling scalable computation, and allowing in-situ interactive analytics.}

% \stitle{Data Model and Algebra.}
% \sajreview{To the best of our knowledge, an algebra for \vita has never been defined previously. Our work takes inspiration from existing algebras that provide well-founded semantics for modeling the data domain of relational databases~\cite{codd}, dataframes~\cite{modin,lara}, and visualizations~\cite{bostock2009protovis,stolte2002polaris,bostock2011d3,satyanarayan2016vega,satyanarayan2015reactive}. A number of grammar of visualizations also capture interactions on data~\cite{satyanarayan2016vega,satyanarayan2015reactive}. \vta introduces both a grammar of \vita view specifications and interactions.}