\section{Discussion}
\label{sec:discussion}
% \sajreview{We now outline our vision for \vita framework development.}

\todo{Discussion and implications based on study results. drop DB specific stuff.}

\todo{lines of code spent in explore and analyze phase \url{https://arxiv.org/pdf/2008.12828.pdf}}

\todo{many data scientists use off the shel models. at least early on.}

\todo{requirement of debugging, cite NBSAFETY}

We now outline our vision for \system development.

\stitle{Coverage, accessibility, and automation.} 
Our goal is to increase \system's coverage of \vita workflows~\cite{liu2018bridging} by introducing new \vta operators and adding popular ML and NLP libraries~\cite{mlbazaar} as default operators in Operator View. We can further improve \system's extensibility by enabling users to add their custom-built models as new operators in Operator View and reuse later. To make \vta more accessible to a wider audience, we are working on integrating \vta with an interactive widget that allows users to issue \vta operators from Jupyter notebooks. Other goals include 
automatically generating \vita workflows given an analysis goal~\cite{bar2020automatically}, recommending next operator based on users' current workflow~\cite{yan2020auto}, and training
an autoregressive language model like \emph{GPT-3}~\cite{brown2020language} on \vta to automatically compose coordinated views or \system-style user interface based on user specifications in natural language.

\stitle{Interaction at scale.}
\system should provide interactive responses even with larger datasets. One approach is to
draw samples of data progressively and then display approximate results. 
Approximate query processing techniques can support operations
like \code{aggregate}~\cite{babcock2003dynamic, agarwal2013blinkdb,acharya1999aqua} and \code{visualize}~\cite{rahman2017ve,hellerstein1997online}.
However, providing meaningful intermediate results progressively for operations like classification
or clustering is challenging---how to
determine model meta-parameters without
scanning the complete data? 
Progressive computation can be complemented by \emph{optimistic analytics}~\cite{moritz2017trust}, where precise computations run on the background as users explore approximate results. When there is a significant difference between
the approximate and precise results (\eg classification results vary from ground truth), the analyst can decide
which parts of the exploration have to be redone.
Larger datasets also impede direct data manipulation and necessitate the design of interactive and navigable representation of Table View~\cite{bendre2019faster}.
%As users edit the data, for example, to relabel hundreds of instances, how can we retrain a model interactively?

\stitle{Versioning \vita sessions.}
\todo{Versioning in the context of data analysis---keeping track of sessions.}

The \system versioning system can maintain a version graph to keep track of fine-grained changes at the unit operation level. 
\system needs to consider the storage-latency trade-off as a user adds new nodes to the graph: storing entire data ensures faster session reconstruction at the cost of storage while storing delta between subsequent session reduces storage overhead at the cost of increased reconstruction time. 
Designing a fine-grained version control system for \system offers unique research challenges---besides \vitaframe, \system also needs to checkpoint
(a) the states of all the front-end components (\eg formatting like font, color, opacity of views), 
(b) the coordination mappings of views and composite selections, (c) user-defined operator pipelines and custom models, and (d) \vta commands in Notebook View. Existing systems address some of these challenges in isolation (\eg data~\cite{huang2017orpheusdb} and model~\cite{miao2016modelhub} versioning, workflow debugging~\cite{brachmannbyour,miao2016provdb}).



\stitle{Data management for \vita.}
Prior work focuses on designing systems for scalable computation (\eg scalable dataframe and query/operator optimizers~\cite{modin}, caching and prefetching for visualization~\cite{taokyrix}), storage models for efficient data access~\cite{tiledb,raasveldt2020data}.
We discussed related work on versioning~\cite{huang2017orpheusdb,miao2016modelhub,brachmannbyour,miao2016provdb}, approximate query processing~\cite{babcock2003dynamic, agarwal2013blinkdb,acharya1999aqua,rahman2017ve,hellerstein1997online} in earlier sections. \system builds on the earlier work with specific focus on developing an efficient storage model, enabling scalable computation, and performing fine-grained version control. \todo{Discuss how \system is different from a data management system. It's more of a study of an end-to-end tool built on top of these systems.}








