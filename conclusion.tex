
\todo{What did we learn from developing Leam so far?} 
\section{Conclusion}\label{sec:conclusion}
This paper presents \system, an integrated system that supports \vita workflows end to end. \system is designed based on several design considerations that we derived by identifying existing challenges in developing \vita systems. Using \system, users can perform interactive text analysis in-situ---direct data manipulation (Table View), REPL-style analytics (Notebook View), and coordinated visual data exploration (Visualization  View).
We build \system on a visual text  algebra,  providing  a suite of operators to author diverse \vita workflows on demand and enable different modes of interactive coordination among views.  We evaluate \system through two case studies using two Kaggle text analytics workflows and find that $\ldots$. 

% This paper presents our vision for \system, an integrated system that supports \vita workflows end to end. \system is designed based on several design considerations that we derived by identifying existing challenges in developing \vita systems. Using \system, users can perform interactive text analysis in-situ---direct data manipulation (Table View), REPL-style analytics (Notebook View), and coordinated visual data exploration (Visualization  View).
% We introduce a novel algebra for visual text analysis, \vta, that provides a suite of operators to author any \vita workflow on-demand and enable different modes of coordination among views.
% We present our current progress in developing \system's underlying data management system and outline several research directions related to 
% \vta extensibility and coverage, and storage, computation, and versioning of data and \vita workflows. Addressing these challenges requires interdisciplinary research efforts from DB, NLP, HCI, and VIS communities.

% \section{Conclusion}\label{sec:conclusion}
% \sajreview{In this paper, we outline our vision for \system, a system that captures the end-to-end flow of \vita. Using \system, users can perform  iterative/nonlinear analysis in-situ---direct data manipulation (\emph{table view}), REPL-style analytics (\emph{notebook view}), and coordinated visual data exploration (\emph{summary view})---without having to move back and forth between various tools.
% We also identify  a number of  challenges related to data management for \vita, \ie 
% efficient storage and computation models for heterogeneous data management and analysis, 
% provenance for 
% workflow and data re-use/reproduciblity, and
% an expressive grammar for \vita workflow abstraction and optimization. 
% We present our current progress 
% in developing \system's underlying data management system 
% and outline the key challenges that such a system should address next. These challenges are not only of interest to the DB community, but also require interdisciplinary research efforts spanning natural language processing, visualization, and human computer interaction.}